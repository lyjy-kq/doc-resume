% !TEX TS-program = xelatex
% !TEX encoding = UTF-8 Unicode
% !Mode:: "TeX:UTF-8"

\documentclass{../../styles/resume}
\usepackage{../../styles/zh_CN-Adobefonts_external}
\usepackage{../../styles/linespacing_fix}
\usepackage{indentfirst}
\usepackage{cite}
\usepackage{setspace}
\onehalfspacing

\begin{document}

\pagenumbering{gobble}

\begin{center}
  \begin{minipage}{\textwidth}
    \centering
    \name{孟高翔}
    % \vspace{0.1em}
    \basicInfo{
      \email{1514190951@qq.com} \textperiodcentered\ 
      \phone{(+86) 136-7557-3771}
    }
    \basicInfo{
      {男 | 安徽泗县 | 中共党员}
    }
  \end{minipage}
\end{center}

\begin{tikzpicture}[remember picture, overlay] 
\node[anchor=north east, inner sep=0pt] at ([xshift=-2.5cm, yshift=-0.7cm]current page.north east) { 
\includegraphics[width=1in]{../../assets/images/photo.jpg} }; 
\end{tikzpicture}


%--------------------------------------
\section{\faGraduationCap\ 教育背景}
% \datedline{\textbf{武汉大学(双一流A类,985高校)|计算机学院|软件工程专业}}{\textit{硕士}\quad \textit{2026.09 -- 2028.06}}
\datedline{\textbf{武汉大学(双一流A类,985高校)|计算机学院|软件工程专业}}{\textit{本科}\quad \textit{2022.09 -- 2026.06}}
\begin{itemize}[leftmargin=*, labelsep=0.5em]
\item \textbf{本科绩点}:3.883\quad \textbf{专业排名}:16/218(7\%)
\item \textbf{荣誉奖项}:校级乙等奖学金(前30\%)、三好学生、优秀学生干部、甲等新生奖学金
\item \textbf{核心课程}:计算机网络(96),数据结构(92),操作系统(91),计算机组成与原理(90),软件需求与建模(90),云计算(92),线性代数(96)
\end{itemize}

%--------------------------------------
\section{\faUsers\ 实习经历}
\datedsubsection{\textbf{腾讯公司-} | 后台开发工程师}{2022.09 -- 2026.06}
\role{(腾讯控股有限公司)}{}
\begin{itemize}[leftmargin=*, labelsep=0.5em]
  \item \textbf{负责工作}:实习期间工作主要集中在xxx
  \item \textbf{工作收获}:……
\end{itemize}


% Section: Project Experience
\section{\faProjectDiagram\ 项目经历}

% Project 1: Star-SaaS Short Link
\datedsubsection{\textbf{星辰Star-SaaS短链接(个人项目)}}{2024.01 -- 2024.02}
\role{后端开发}{SpringBoot, SpringCloudAlibaba, RocketMQ, ShardingSphere, Redis, MySQL, Sentinel}
\begin{itemize}[leftmargin=*, labelsep=0.5em]
  \item \textbf{项目介绍}:SaaS短链接系统,提供高效、安全、可靠的短链接管理平台,简化长链接的管理与分享过程,并提供深入的分析与跟踪功能,助力用户灵活管理链接以提升营销效果。
  \item \textbf{线上地址}:\url{https://horiik} \quad \textbf{仓库地址}:\url{https://github.com}
  \item \textbf{功能描述}:
    \begin{itemize}
      \item 采用布隆过滤器快速判断短链接是否存在,性能远超分布式锁结合数据库查询的方案。
      \item 利用RocketMQ消息队列的削峰填谷特性,支持海量访问场景下的监控信息存储,实现存储与访问解耦。
      \item 通过更新数据库后删除缓存的策略,确保缓存与数据库之间的数据一致性。
      \item 基于Redis实现消息队列消费的幂等场景,保障消息在一定时间内仅被消费一次。
      \item 结合读写锁与RocketMQ延迟队列,支持短链接分组在高并发场景下的数据修改。
      \item 使用Sentinel实现接口QPS限流,触发限流规则后进行降级处理,确保系统稳定运行。
    \end{itemize}
  \item \textbf{项目收获}:通过开发本项目,深入理解并发场景下的开发设计,对SaaS架构、微服务、网关及流量控制有深刻实践经验,提升了分布式系统开发能力。
\end{itemize}
% Project 2: Star-RPC Framework
\datedsubsection{\textbf{星辰RPC框架(个人项目)}}{2024.03 -- 2024.04}
\role{系统开发}{Spring, Netty, Kryo, Zookeeper, Guava; C++11: ReactorTCP, Protobuf, RPC, TinyXML}
\begin{itemize}[leftmargin=*, labelsep=0.5em]
  \item \textbf{项目介绍}:星辰RPC框架,模拟实现了一个基础RPC框架,支持注册中心、网络传输、序列化、动态代理及负载均衡功能,同时提供传统BIO方式的Socket传输及JDK序列化作为对比方案。
  \item \textbf{仓库地址}:\url{https://github.com}
  \item \textbf{功能描述}:
    \begin{itemize}
      \item 使用Zookeeper作为注册中心,负责服务地址的注册与查找。
      \item 基于Netty实现RPC网络请求传输,包含调用的类名、方法名及参数。
      \item 采用Kryo序列化机制,优化性能并解决JDK序列化效率低及安全漏洞问题。
      \item 实现特定负载均衡策略,避免单服务器响应同一请求导致宕机或崩溃。
    \end{itemize}
  \item \textbf{项目收获}:通过开发本框架,掌握了RPC架构设计与实现,深化了对分布式系统、网络传输及服务治理的理解。
\end{itemize}
% Project 3: Axin-Star-Admin
\datedsubsection{\textbf{Axin-Star-Admin(个人项目)}}{2023.07 -- 2023.12}
\role{全栈开发}{SpringBoot, MyBatis, JWT, Redis, MySQL, Vue, TypeScript, Docker, Git}
\begin{itemize}[leftmargin=*, labelsep=0.5em]
  \item \textbf{项目介绍}:前后端分离的管理系统,基于RBAC模型实现细粒度权限控制,涵盖用户、角色、菜单管理,并扩展了Blog和Game管理模块,优化个人网页发布流程。
  \item \textbf{线上地址}:\url{https://admin} \quad \textbf{仓库地址}:\url{https://github.com}
  \item \textbf{功能描述}:
    \begin{itemize}
      \item 基于RBAC模型实现接口及按钮级别的细粒度权限控制,支持用户、角色、菜单管理。
      \item Blog模块支持文章、分类、标签管理,简化网页发布流程,定时任务存储草稿避免丢失,图片上传至阿里云OSS图床。
      \item 支持一键发布文章至Hexo前台目录,调用shell脚本重新部署并刷新CDN。
      \item 采用阿里云STS临时凭证访问OSS及CDN,凭证缓存至内存,使用线程池定时清理过期缓存。
      \item 编写Dockerfile制作Java后端镜像,结合docker-compose集成MySQL、Redis、Nginx及OpenJDK镜像,实现一键部署所有容器。
    \end{itemize}
  \item \textbf{项目收获}:通过开发本系统,优化了项目统一管理能力,深入实践后端主流技术栈及前后端分离开发流程,提升了全栈开发与容器化部署经验。
\end{itemize}
%--------------------------------------
% 定义科研与竞赛经历部分
\section{\faTrophy\ 科研与竞赛经历}
\datedline{\textit{省级大创—智能招聘信息匹配平台 | 核心成员}}{2025.03 -- 2025.06}
\datedline{\textit{2024年全国大学生数学建模竞赛湖北赛区三等奖 | 核心成员}}{2024.09}
\datedline{\textit{第九届华为ICT大赛编程赛省赛一等奖 | 核心成员}}{2025.04}
\datedline{\textit{第十五届蓝桥杯湖北赛区二等奖 | 个人比赛}}{2024.04}

%--------------------------------------
\section{\faInfo\ 其他信息}
\begin{itemize}
  \item 英语水平:雅思6.5、CET-4、CET-6
  \item 专业技能:熟悉 Java、C++、Go、TypeScript、Python;熟悉常见前后端框架;熟悉 MySQL、Linux、Git、Maven、Docker、K8s 等
  \item 兴趣爱好:体育竞赛(院运会篮球赛铜牌、排球赛第五名)、篮球、游泳
\end{itemize}

\end{document}
